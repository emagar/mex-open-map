%\documentclass[letter]{article}
\documentclass[letter,12pt]{article}
\usepackage[letterpaper,right=1.25in,left=1.25in,top=1in,bottom=1in]{geometry}
\usepackage{setspace}

\usepackage{hyperref}
\usepackage{url}
\usepackage[pdftex]{graphicx}
\usepackage[longnamesfirst, sort]{natbib}\bibpunct[]{(}{)}{,}{a}{}{;}
\usepackage{arydshln}
\usepackage{amsmath}

%\usepackage{rotating}
%\usepackage{pdflscape}

\newcommand{\mc}{\multicolumn}

\graphicspath{{../graphs/}}

\begin{document}

\title{Malapportionment and representation: \\ party bias and responsiveness in Mexico\thanks{Paper prepared for the Analyzing Latin American Politics conference, University of Houston, Nov.~14, 2014. We thank the Federal Electoral Institute (IFE) and the Federal Voter Registry (RFE). Special thanks to IFE's Cartography Department for sharing their experience with automated redistricting since 1996 and most of the data we analyze. We also acknowledge the support from the Asociaci\'on Mexicana de Cultura A.C.\ and CONACYT for this work.}}
\author{Micah Altman {\small \url{escience@mit.edu}} \\
        Eric Magar {\small \url{emagar@itam.mx}} \\
        Michael McDonald {\small \url{michael.mcdonald@ufl.edu}} \\  %\\ University of Florida, Gainesville\\
        Alejandro Trelles {\small \url{lat44@pitt.edu}}
      }
\date{\today}
\maketitle


\begin{abstract}
\noindent Automated redistricting in consultation with parties since 1997 rid redistricting in Mexico of the most flagrant distorsions on representation. But more subtle ones remain. Rules and practice introduice some degree of malapportionment, small in comparative perspective \citep{samuels.snyder.2001}, but nonetheless substantial. We inspect district maps, votes, and seats in Mexico in search for two forms of political distorsion that may arise from malapportionment: party bias and responsiveness (otherwise known as plurality bias). State-level data reveal no evidence of systematic party bias. What bias there is is of another kind, a big bonus for larger parties (or vote responsiveness) typical of plurality rule in single-member districts. Since most states distribute few seats each, the large party bonus of states' federal districts doubles the estimate with national data. Since the PRI is the largest party in most states, it receives a substantial seat bonus.
\end{abstract}

% Comment [Eric]: Possible argument for the presentation.
% Expert committees can remove partisan bias from redistricting.
% Mexico's process allows parties to respond. Offers leverage to infer party preferences and degree of influence. 
% Analysis of party bias and responsiveness informs this, directly. 

\onehalfspacing

Mexico has done population censuses every decade since 1920, and every five years since 1990. The relatively short inter-censal periods offer relatively precise population estimates of boroughs in every corner of the country, despite rapid demographic shifts. This official and public data is priceless for planners of all stripes, marketers, and researchers, to name a few. 

But not everyone has valued the data. Redistricters in the last decades are among them, routinely shunning population data availability. Part of their reluctance finds explanation in a Constitution mandating the use of decennial censuses, not lustrum population counts, to do their job; with no attempt to estimate population change since the census is published; and, unlike the U.S., no obligation to redistrict as soon as a new census is available. But much of the blame falls on legislators, parties, and electoral regulators in charge of redistricting, who have, to larger or smaller extent, but systematically lagged behind the constitutional obligation to ensure equal representation when a census becomes available. 

To get an idea of this, consider that information used to draw a new map for the 1979 congressional was nearly a full decade old upon the map's inauguration. The rural exodus was at its peak then. That same map had missed nearly 25 years of demographic change when it was last used to translate votes into seats, in 1994. Likewise, the map that replaced it was drawn with six year old information. Had the lustrum count been used instead of the general census, that lag could have been a single year. And when the 2015 midterm congressional election takes place, a decade-and-a-half gap will separate the map and the actual district populations. 

Malapportionment arises as a consequence. Rural districts lose population to urban districts. (An intuition that needs development in the paper: the PRI remains much stronger than other parties in the countryside \citep{amesMex.1970,magar.1994}, so this malapportionment may play in its favor. True? Has PRD also entered the fray?)

This paper inspects legislative malapportionment in Mexico, the developments that have led to it, and some of its consequences for representation. After describing the redistricting process as it has been implemented since 1997, the paper describes two forms of malapportionment and how prevalent they are in Mexico. Attention then turns to a model estimating two common distorsions of electoral systems: party bias, and responsiveness (or large party bias). 

\section{Redistricting since 1997}

Mexico adopted a mixed-member electoral system for the lower house of Congress for the 1979 congressional election. Three hundred single-member districts were drawn to elect three-fourths of the chamber by plurality rule. That proportion fell to three-fifths since 1985, but the number of single-member districts has not changed. No boundary lines are drawn for the compensatory, proportional representation seats; accordingly, we leave them out of the analysis, and by seats/districts, in the paper, we always refer to the single-member kind. The original district map was redrawn in 1997, coinciding with the first free and fair congressional election \citep{lujambio.vives.2008}. Redistricting occurred again in 2006, and was meant to be undertaken, but aborted, in 2015. 

We use years to name the different maps, sticking to the convention of using not the date when boundaries were actually redrawn and accepted, but that of the first election the map was (or would have been) used. Thus, there is a ``1979 map'' actually drawn in 1978, a ``1997 map'' drawn in 1996, a ``2006 map'' drawn in 2004--05, and a ``2015 map'' drawn in 2013, but never used.

The method adopted has relied on machine-assisted mapping since 1997. Details have changed, but the broad lines have not \citep{trelles.mtz.tesisItam.2007}. We describe the three major stages of the redistricting process: the apportionment of seats to states; the design of an optimization algorithm; and the strategic assessment of computer-generated district blueprints by a technocrats with active, but limited involvement of political parties. 

\textbf{Apportionment}. This stage redistributes a fixed number of seats among the 32 states.\footnote{The Federal District, where Mexico City resides, is not a state proper, but is treated as one for the purpose of apportionment and redistricting.} Restrictions, reminiscent of the U.S., apply: no state may have fewer than two seats; the sole redistributive criterion above that minimum is state population; and no district may cross state boundaries. The next section elaborates the apportionment method used and problems this raises for representation. 

\textbf{Optimization}. This stage adopts a set of desiderata for a new map drawn from the base geography. The process makes marginal changes to boundaries, repeatedly reassigning geographical blocks between districts, systematically checking tradeoffs between criteria in a formal cost function, and iterating it all with different random seeds. Cost-minimizing district maps for each state are thus generated. The building blocks are more than 66 thousand \emph{secciones electorales} into which the Mexican territory is subdivided. \emph{Secciones} are analogous to U.S.\ census tracts, but somewhat bigger. Median \emph{secci\'on} population in the 2010 census was 1,280 persons, with a maximum at 79,232. Restrictions also apply: contiguity is a must, no district can have exclaves; as are guarantees for minority representation, \emph{municipios} (the lowest elected offices, similar to counties in the U.S.) with sizeable indigenous populations cannot be partitioned. 

More refined optimization algorithms have been used in each redistricting round. The algorithm described in the last paragraph, used for the 2015 map, replaced a method of simulated annealing used for the 2006 map. The most recent cost function included four criteria (with varying relative weights): population balance (.4), municipal boundary preservation (.3), travelling times inside the district (.2), and district compactness (.1). IFE would tolerate population deviations of up to 15\% from mean district size (10\% in 2006), and deviations beyond that threshold with proper justification. A tecnical committee, appointed by IFE's Council General, is in charge of optimization to propose a map for assessment. 

\textbf{Assessment}. In this stage, that we study elsewhere \citep{altman.magar.mcd.trelles2014apsa}, the technical committee, party representatives, and the Council General interact. The map proposal is distributed to the parties for analysis and discussion. Parties suggest modifications to district boundaries, which the technical committee discards or accepts by means of the optimization algorithm. Those that are adopted become the starting point for a second and final round of observations/amendments. The final plan is presented to the Council General to be formally approved. It is at this stage that the most recent redsitricting process was abandoned. The Council General opted to shelve the plan until the details of an ongoing major electoral reform in Congress became known. As a consequence, the 2015 midterm election will be held with significantly outdated districts. 

\section{How to read a map}

[Digression. We may drop it.]

A new map is a full description of district boundary changes. The common way to visualize maps is on paper. We look at them differently, with focus not in the physical lines and their relation to geographic accidents, but in their political consequences. Putting the 2006 map and the 2015 final proposal drawings side to side reveals boundary shifts. But unless the drawing is more data rich, assessing even the relatively simple question of how similar districts were before and after redistricting visually is a challenge. 

\begin{table}
\begin{center}
  \begin{tabular}{lrrrrr}
  Similarity between                 &   min  &  25\%  & median &  75\% &  max \\ \hline
  first 2015 proposal and status quo & 0.128  & 0.419  & 0.584  & 0.755 &  1   \\
  final 2015 proposal and status quo & 0.125  & 0.437  & 0.643  & 0.805 &  1   \\
  first and final 2015 proposals     & 0.174  & 0.705  & 0.967  & 1     &  1   \\
  \end{tabular}
  \caption{District similarity before and after the failed 2015 redistricting}\label{T:simIndex}
\end{center}
\end{table}

But a similarity index \citep[][:15--7]{cox.katz.2002} offers ground for assessment. Measurement identifies the parent district or single largest contributor of population to every new district. The $s$imilarity index divides the population that $p$arent and $n$ew district share in $c$ommon by their joint population: $s = c / (p + n - c)$. The range is zero (district shares no population at all with parent) to one (districts are identical). Table \ref{T:simIndex} describes the change that the 2015 map would have brought in the 300 single member districts, comparing IFE's initial and final proposals with the 2006 map. Also compared are the final and initial 2015 proposals themselves. 

Six percent of districts entertained no change in boundaries whatsoever vis-\`a-vis the status quo (18 districts in the initial proposal, 19 in the final), and twice as many had at least nine-tenths of the population in common (40 districts in the initial, 44 in the final proposal). Inspecting which districts were \emph{a priori} chosen for marginal or no change might reveal systematic distortions of the automated redistricting algorithm. Party feedback left 39 percent of districts in IFE's first proposal intact, and 62 percent with up from nine-tenths of the population in common. Inspecting which districts the parties targeted for major amendment --- presumably a partial reinstatement of a preferred status quo district --- should shed further light into the subject. Note how median district similarity with the 2006 map augmented from first to final proposal. So did the third quartile. Accepted counter-proposals therefore reconstituted portions of status quo districts. This seems consistent with a view where parties protect strongholds that the algorithm may have split, but further analysis must be done. 

[End digression.]

\section{One person, one vote?}

Of many ways in which malapportionment can arise \citep{snyder.samuelsMalapp2004}, two are of interest here: that ensuing from the distribution of seats between the states, and that ensuing from drawing district boundaries within the states. We discuss them in turn.

\subsection{Between states}

In sharp contrast to the U.S., little, if any public debate about alternative apportionment methods has been undertaken \citep{szpiro.numbersRule.2010,balinski.rodriguez.1996}. The Hamilton method of largest remainders is used for apportionment of Mexico's national legislative seats to the states \citep[][:10]{balinskiYoung2001FairRep}. The quota $Q$ (or price) of a seat is the nation's population divided by 300, the number of seats. A first allocation is made by dividing each state's population in the last general census by $Q$, rounded down. States with populations $<2Q$ are nonetheless allocates two seats (Colima and Baja California Sur fell in this category in early apportionment rounds). Unallocated seats, if any, are awarded to states with largest fractional remainders. 

\begin{figure}
\begin{center}
  \begin{tabular}{cc}
    \includegraphics[width=.45\columnwidth]{statesUnderOverRep.pdf} & 
    \includegraphics[width=.45\columnwidth]{statesUnderOverRep-rel.pdf} \\ 
  \end{tabular}
  \caption{Demography and state apportionment. Lines connect the lower chamber seats aportioned to seats due difference for each state over time. Letters R in the horizontal axis indicate redistricting, letter F a failed redistricting attempt (reporting the effect it would have had).}\label{F:underOverRep}
\end{center}
\end{figure}

Figure \ref{F:underOverRep} summarizes state's representation in Congress. We estimated each state's population at the time of each federal election since 1997. Estimation between 1997 and 2010, inclusive, relied on linear interpolation of the lustra census populations,\footnote{The years 1995 and 2005 had population counts, 2000 and 2010 had general censuses. The former collect much fewer attributes than the latter.} and linear extrapolation for 2012 and 2015. With moving population estimates, a time-series of states' fair share of congressional seats was computed (time-varying $Q$s). The actual number of seats apportioned is subtracted, yielding the state's over- or under-representation at the time of the election. A state with perfect apportionment in every election would appear as a flat, horizontal line at zero in the Figure. The left panel reports this measure in absolute terms (so that $+5$ indicates a state with five deputies more than its fair share), the right panel relative to seats apportioned to the state ($+5$ indicates five percent more deputies than its fair share).  

Since the first term of the subtraction is a real number and the second an integer, a fraction is bound to remain. Fractions in well-apportioned systems should all be less than 1 in absolute value. The left panel shows that this is mostly the case for Mexico, but exceptions are systematic. It is also plain that redistricting towards 2006 corrected, to some extent, important distortions that redistricting towards 1997 had failed to rectify. Under-representation of Baja California, but especially of the Mexico State dropped respectively from two and five deputies less than due (deficits of 30 percent and less than 10 in relative terms), respectively. Over-represented Veracruz behaved near symmetrically to Baja in absolute terms (not relative). And a distortion strongly favoring the Federal District ($+4$ deputies) was somewhat attenuated towards the 2006 election, but never removed. Relative over-representation is substantial for the smallest states---entitled to two seats at least like all states, which Southern Baja California and Colima weren't really worth demographically until quite recently---migrant-worker-exporter Zacatecas up to 2003, and the Federal District of late. With this brief description, it is puzzling that the redistributive nature of apportionment has not pushed for the adoption of alternative methods. The seventeen states that were under-represented in 2006 jointly controlled a majority (162) of single-member districts; fourteen of them have always been below the red line indicating fair representation. 

%``Natural malapportionment'' is how \citet{snyder.samuelsMalapp2004} call demographic drift.

%Over- and under-representation vis-\`a-vis demography is one form of malapportionment that few electoral systems can render negligible. Mexico's does not. 

\subsection{Within states}

Despite automation and straightforward, formal redistricting criteria, Mexican parties in general, and IFE in particular, have been remarkably tolerant to unequally sized districts --- another form of malapportionment. It is related in some degree to distortions last discussed because differences in state apportionment perforce create size differences across states. But size inequality within states is also prevalent and substantial. 

Small deviations around a state's mean district population are unavoidable. But what constitutes a small deviation remains a matter of opinion. Courts in the U.S.\ have struck down district maps bearing less than 1\% differences without proper justification \citep{tuckerApportionment.1985}. Redistricting authorities generally view a \emph{de minimus} population deviations of as little as one or zero persons between congressional districts as desirable to inoculate against litigation. In stark contrast, when redistricting, IFE considered deviations between 10 and 15\% above or below mean state district size perfectly normal. Within such spread, a district at the bottom end is worth one-third more in Congress than one at the top end. Surprisingly, no party has ever challenged this in Court. 

\begin{figure}
\begin{center}
    \includegraphics[width=.7\columnwidth]{malapp2006d0.pdf} \\
    \includegraphics[width=.7\columnwidth]{malapp2015d0.pdf} \\
    \includegraphics[width=.7\columnwidth]{malapp2015d3.pdf} \\
  \caption{Malapportionment through years and maps. The $y$-scale measures district population relative to ideal (mean state population). The grey band is IFE's range of tolerance for district population asymmetries.}\label{F:malapp}
\end{center}
\end{figure}

Mexico's redistricting automation makes an attempt for districts within states to tend towards zero malapportionment, but this criterion is often trumped by others in the algorithm. The preservation of municipal boundaries, for instance, is achieved by exploiting tolerated leeway. Figure \ref{F:malapp} shows how the tolerance band is uniformly occupied by districts even right after inception. It is notable that the new 2006 map had so many districts outside the range of tolerance. 

When redistricting plans are drawn, malapportionment does not exist \emph{de jure}. It often exists \emph{de facto}. Mexico's constitution provides that every redistricting process must rely on the most recent population census available, with no attempt to estimate population change since (and, unlike the U.S., no obligation to redistrict as soon as a new census is available). By 2006, population changes from 2000 to 2005 resulted in the mean state population that year deviating 9.7 percent compared to the census, with a standard deviation of 6.9 percent. Variance cannot be smaller when dealing with sub-state units used to prepare districts, pushing about 80 new districts out of the broad $\pm 15$\% band.\footnote{Census data reported at the electoral \emph{secci\'on} level are available since 2005 only. We could repeat this exercise at the municipal level to approximate distortions that the law pushed IFE to introduce better.} The requirement to keep municipalities with large indigenous population within the same district may also explain exceptions. Median absolute district malapportionent of the 2006 map upon inauguration was 6.5\%, the third quartile surpassing the ten point tolerance margin at 12.4\% \citep[also,][]{trelles.mtz.tesisItam.2007}. For next year's midterm election, the same descrepancies will be 9.9 and 17.9\%, respectively. The abandoned 2015 map proposal did a better job upon inauguration, with median absolute district malapportionment of 4.2\% and third quartile at 8.1\%. 

\begin{tabular}{lrrrrr}
                 & \mc{5}{c}{Absolute malapportionment} \\
Map in year      & Min. & 1st quartile & median & 3rd quartile & max. \\ \hline
2006 map in 2006 & $<.01$ & 3.4 & 6.5 & 12.4 & 54.8 \\
2006 map in 2015 & .02 & 4.4 & 9.9 & 18.0 & 147.2 \\
2015 map in 2015 & .01 & 1.9 & 4.2 & 8.1 & 50.6 \\
\end{tabular}

Whether districts below and above fair representation behave differently merits closer inspection. Whether due to technical difficulties or discretion, malapportionment may be the source of meaningful distortions in democratic representation. To which we now turn to. 

\section{Party Bias and Responsiveness}

This section estimates district responsivity and party bias---two effects of scholarly interest, discussed below. M\'arquez (\href{http://bit.ly/1jgsQXE}{\url{http://bit.ly/1jgsQXE}}) has also approached this with national aggregates, uncovering a degree of responsivity characteristic of Westminster systems and party bias against the PAN under the status quo. By proceeding with state aggregates, much higher responsivity (owing to fewer districts, 9 by state on average) is uncovered, but no apparent party bias in neither the 2006 map nor 2015 proposals. 

\subsection{Systematic Bias}

\begin{figure}
\begin{center}
    \includegraphics[width=.6\columnwidth]{resXedo20062012.pdf} 
\caption{Seats and votes in the states. Each point is a party-state-year, blue for PAN, red for PRI, gold for PRD, other colors for minor parties.}\label{F:seatsVotes}
\end{center}
\end{figure}

Consider now the relation between party votes and seats portrayed in Figure \ref{F:seatsVotes}. As before, the plot considers plurality seats only and state aggregates. Each point reports the vote share that a party won in a state's federal deputy elections (x-axis) and the share of the state's congressional seats it received (y-axis). Colors distinguish the parties: PAN is blue, PRI is red, PRD is gold, Green is green, among others. For instance, the green dot floating to the left of the cloud is the Green party in Chiapas 2012, where it won 3 of 12 districts. The chart shows that 9 percent of the state's vote awarded the party 25 percent of the seats, an outstanding achievement for any party. The cloud manifests a steep upwards slope characteristic of first-past-the.post systems \citep{taagepera.CubeLaw.1973}.\footnote{Adding the excluded PR seats would level the slope considerably. Doing this would be easy with national aggregates. It is not evident how to carry it with state aggregates, since PR seats are awarded in five second-tier districts joining together several states each.} Points below the diagonal indicate under-representation, those above over-representation. There are notable differences among major parties: the PRI achieved over-representation in three-fifths of election-states between 2006 and 2012, the PAN in two-fifths, and the PRD in one-fourth only. 

With such setting, the possibility that districts are granting undue advantage to the PRI merits closer inspection. A priori, reasons to suspect IFE of cooking districts to favor one party or another are lacking. Major parties, after all, permanently influence the election regulator, ambition counteracting ambition \citep{estevez.magar.rosas.2008}. But that those who draw the district lines can distort a fundamental link of the democratic process is well established in the literature \citep{altman.mcdonald2011bard,cox.katz.2002,engstrom2006redisttrictApsr,rossiter.etal.1997,king.1990elRespBiasMultiparty,balinskiYoung2001FairRep,otero.2003}. Has the insidious gerrymandering reared its ugly head in Mexico? This section estimates majority and party effects in the proposed districts and the status quo. Majority effects are huge, partisan effects negligible. 

\subsection{Two Classes of Distortion}

Undue advantage is known, in the specialized literature, as partisan bias, and is one goal that strategic redistricters pursue. It is not, however, alone: scholarship highlights district responsiveness, also know as majoritarian bias, as another goal. These ought to be distinguished \citep[this paragraph draws heavily on][, ch.\ 3]{cox.katz.2002}. Partisan bias helps the beneficiary buy seats with fewer votes than others. Because seat distribution is a constant sum game, bias in favor of someone always implies bias against someone else. One way of introducing party bias in district lines is with the conventional redistricting strategy known as packing: group your adversary's voters in few districts, wasting votes to win unnecessarily safe seats, raising the price of victory. Responsiveness, on the other hand, is the feature granting a seat bonus to large parties. Maximal responsiveness occurs within each single-member district in isolation: the winner takes all, the rest nothing. The same could be achieved in a whole state by drawing lines so that every district is representative of the state's electorate (Cox and Katz's microcosm strategy). The party with most votes wins every seat, maximizing the vote responsiveness of the proposal. 

Formalizing party bias and responsiveness opens the way towards estimation of these district characteristics. The two-party case is simpler and extends to multiparty systems \citep{taagepera.CubeLaw.1973,tufte1973seatsVotes,king.browning1987biasRespUS}. It is a generalization of the cube law stipulating that 

\begin{equation}
 \frac{s}{1-s} = e^\lambda *  \left(\frac{v}{1-v}\right)^\rho \iff
 \texttt{logit}(s) = \lambda + \rho *  \texttt{logit}(v)
\end{equation}\label{E:cubeLaw}

\noindent where $s$ is the seat share that party 1 won with vote share $v$; $\lambda$ is party 1's bias relative to party 2 (positive values favor party 1, negative values favor party 2); and $\rho$ is the districts' responsiveness. With $\lambda=0$ a system with no party bias ensues. Figure \ref{F:lambdaRhoEx} shows how the parameters affect the vote-to-seats conversion. 

% comment [Mike] how do multiple parties affect bias/responsiveness estimates, compared to two-party setting? I would expect that a party would often need less than 50% of the vote to win 50% of seats due to many parties receiving votes in single-member districts.

\begin{figure}
\begin{center}
    \includegraphics[width=.55\columnwidth]{rhoExample.pdf} 
\caption{Illustration of estimated parameters. Party bias is set to $\lambda=0$ in non-grey lines. Grey lines replicate the colored ones with $\lambda=+1.5$.}\label{F:lambdaRhoEx}
\end{center}
\end{figure}

Non-grey lines lack party bias to illustrate variable responsiveness. A system with $\rho=1$ is perfectly proportional representation, the ideal type against which the evaluation of real districts are often contrasted. It appears as the dotted green diagonal: every party winning $x$\% of the vote gets, precisely, $x$\% of seats. $\rho=3$ characterizes the classic cube law, the red curve over-representing the winner (points above the diagonal). Here a party with 55\% of the vote earns two-thirds of the seats, but with 33\% it earns only one-tenth of the seats. As responsivity grows, the curve gets steeper, until barely crossing the majority threshold suffices to win all the available seats. 

Grey lines replicate the values of $\rho$ just discussed but with $\lambda = 1.5$ added. Bias in favor of the party produces a leftward pull of lines. In other words, a bias-favored party requires less effort to reach the threshold for large-party over-representation. (The grey dotted line demonstrates how, due to logit links in Equation 1, party bias also reshapes the function's trace.)

A multiparty and estimable version of equation 1 \citep{king.1990elRespBiasMultiparty} establishes that party $j$'s ($j=1,2,\ldots,J$) expected seat share is 
\begin{equation}
 E(s_j) = \frac{e^{\lambda_j} * v_j^\rho}{\sum_{m=1}^{J} e^{\lambda_m} * v_m^\rho}
\end{equation}

\noindent with data and parameters indexed to identify the parties. \citep[Another, with application to Argentine federalism, is][.]{calvo.micozzi.govReform.2005} Setting $\lambda_2 = 0$, as done below, forces the remainder $\lambda_{j \neq 2}$ to express party bias with relation to the PRI's ($j=2$ for this party in the dataset). This is convenient to test the presumption of PRI-favoring bias:. if present, $\hat{\lambda}_{j \neq 2}<0$ would result.

A common estimation strategy relies on a time-series of national aggregates (this is what M\'arquez did for eight elections 1991--2012, uncovering substantive anti-PAN bias and high responsiveness). The estimation strategy here, as before, is with state-level data. One disadvantage of the approach is that states have few districts (9.4 on average), and this will amplify the system's responsiveness \citep{taagepera.CubeLaw.1973}. The advantages are many. The approach multiplies observations. This is evident in Figure \ref{F:seatsVotes}, with many points despite reporting three elections only. It holds the actual district structure constant---redistricting in 1997 and 2006 invalidates district comparability before and after. And it takes advantage in the variation of state party systems (this draft fails to take full advantage of this variance, but this could be exploited further). 

\begin{figure}
\begin{center}
  \begin{tabular}{cc}
    (a) & (b) \\
    \includegraphics[width=.45\columnwidth]{resp200612s0s3.pdf} &
    \includegraphics[width=.45\columnwidth]{bias200612s0.pdf} \\
    (c) & (d) \\
    \includegraphics[width=.45\columnwidth]{bias200612s1.pdf} &
    \includegraphics[width=.45\columnwidth]{bias200612s3.pdf} 
  \end{tabular}
  \caption{Redistricting, responsiveness, and party bias. Plots report the posterior sample of model's parameters $\lambda_j$ for four parties and $\rho$, and the median value (black circles).}\label{F:posterior_s0s1s3}
\end{center}
\end{figure}


The  method of estimation is MCMC \citep{jackman.2000}.\footnote{Three chains were iterated 10 thousand times, taking every fiftieth observation of the last 5 thousand to sample the posterior distribution. Convergence was gauged visually with traceplots of the separate chains for each of the model's parameters. Estimation performed with JAGS \citep{jags.cite}, implemented from R \citep{r.cite} using package R2jags \citep{r.r2jags}. Data and code to replicate the analysis can be found at HTTP.} As expected, district responsiveness is extremely high, between 5 and 6 depending on the year selected (the steepest line in illustrative Figure \ref{F:lambdaRhoEx} has $\lambda=6$). Figure \ref{F:posterior_s0s1s3}.a reports point estimates (the black circles are the median of the posterior sample of the responsiveness parameter) for each year separately, all years pooled together, and comparing actual districts to the first and third redistricting proposals. Estimate precision is assessed with the cloud of grey points (technically, it is the full sample of posterior $\rho$s). The redistricting proposal makes little change to district responsiveness, it just makes it slightly more volatile from one election to next. Owing to few districts per state, the estimated responsiveness at the state level ($\hat{\rho} \approx 5.5$) is twice M\'arquez's nationwide ($\hat{\rho} = 2.6$). All three parties experience situations of large party bonus and small party penalty that, to a good extent, cancel each other in the national statistic.

Regarding party bias, signals that are not weak all tend to be accompanied by a good deal of noise, with few exceptions. At the national level, and over a longer haul, M\'arquez discovers bias in favor of the PRI, but mostly in favor of the PRD, and against the PAN, that seems not the product of chance alone. Analysis at the state-level reveals no such biases. As said, Figures \ref{F:posterior_s0s1s3}.b--d express bias relative to the PRI. Although PAN experienced weak signal in its favor in the whole 2006--12 period, a fair density of the blue cloud is, in fact, negative. PRD vs.\ PRI bias is clearly centered at zero in the full period. The left did experience significant bias in isolated years: against in 2006, in favor in 2009. Perhaps voters who strategically abandoned the hopeless PRI presidential candidate in 2006 to vote for L\'opez Obrador did not also endorse the PRD's congressional candidates. No other year for no other party reveals any bias unaccompanied by much noise.

% \section{Malapportionment and margin}

% \begin{figure}
% \begin{center}
%   \begin{tabular}{cc}
%     \includegraphics[width=.4\columnwidth]{malmg2006d0sta.pdf} & \includegraphics[width=.4\columnwidth]{malmg2006d3sta.pdf} \\
%     \includegraphics[width=.4\columnwidth]{malmg2009d0sta.pdf} & \includegraphics[width=.4\columnwidth]{malmg2009d3sta.pdf} \\
%     \includegraphics[width=.4\columnwidth]{malmg2012d0sta.pdf} & \includegraphics[width=.4\columnwidth]{malmg2012d3sta.pdf} \\
%   \end{tabular}
%   \caption{Vote margin and malapportionment (compared to state averages)}\label{F:malmgsta}
% \end{center}
% \end{figure}

% \begin{figure}
% \begin{center}
%   \begin{tabular}{cc}
%     \includegraphics[width=.4\columnwidth]{malmg2006d0nat.pdf} & \includegraphics[width=.4\columnwidth]{malmg2006d3nat.pdf} \\
%     \includegraphics[width=.4\columnwidth]{malmg2009d0nat.pdf} & \includegraphics[width=.4\columnwidth]{malmg2009d3nat.pdf} \\
%     \includegraphics[width=.4\columnwidth]{malmg2012d0nat.pdf} & \includegraphics[width=.4\columnwidth]{malmg2012d3nat.pdf} \\
%   \end{tabular}
%   \caption{Vote margin and malapportionment (compared to national average)}\label{F:malmgnat}
% \end{center}
% \end{figure}

% Plots in Figures \ref{F:malmgsta} and \ref{F:malmgnat} 

\section{Electoral volatility}

* * To be written * * May help explain why malapportionment has no larger consequences. 

One factor that may explain the absence of party bias is electoral volatility. It may be larger than bias, ecpliping it.

Noting that victory margins began widening in the mid-1960s, \citet{mayhew1974vanishingMg} saw gerrymandering as one possible explanation, incumbents influencing the preservation of safe districts. While the argument opened the heated incumbency advantage debate, the intuition may guide our inquiry. District volatility should capture a key element for analysis of map similarities. District $d$ volatility is $v_d = 1/2 \sqrt{ \sum_{p=1}^P \sum_{t=2}^3 (v_{d,p,t}- v_{d,p,t-1})^2 }$ with $p$ indexing the competing parties and $t=1,2,3$ for the 2006,  2009, and 2012 congressional elections respectively.\footnote{The measure of volatility proposed is inspired in measures of disproportionality, replacing the seats--votes difference with vote first differences. It is a squared version of a Loosemore-Hanby index \citep{loosemore.hanbyDisproportionality1971,gallagherDisproportionality1991}.} 

Mexican party strength is not proportional to the formidable entry barriers they enjoy and massive public subsidies they receive yearly \citep{magar.2007ref.2015}. Evidence of this is their inability to cultivate loyal voters. District volatility is remarkably high, as shown in Table \ref{T:volatMarginsd0}. The median district saw 25 percent of votes change party hands between 2006 and 2012 (not exactly how the index reads, but close enough). With so many volatile districts around, parties should have attempted to redress strongholds that automated redistricting split beyond recognition. How much were strongholds affected by the initial 2015 proposal? To what extent did party counter-proposals redress them? This should be a promising line of inquiry. 

\begin{table}
\begin{center}
\begin{tabular}{lrrrrr}
                    &  min.   &  25\%   & median  & 75\%   & max \\ \hline
district volatility &  .08    & .19     & .25     & .31    & .52 \\
mean PAN margin     &  $-.49$ & $-.23$  & $-.10$  & .01    & .28 \\   
mean PRI margin     &  $-.43$ & $-.09$  & $-.01$  & .07    & .28 \\   
mean PRD margin     &  $-.51$ & $-.31$  & $-.21$  & $-.04$ & .39 \\
\end{tabular}
\caption{2006 map district volatility and party margins in three congressional elections}\label{T:volatMarginsd0}
\end{center}
\end{table}

% comment [Eric] One possibility is regressing new district similarity on mean 2006--12 parent district margin, low volatility, and their interaction. Another route is the estimation of district elasticities---how vote swings in a district's \emph{secci\'ones} respond to statetwide party swings. This could provide an alternative approach to explain the district similarity index. The appendix has preliminary elasticity estimates and some discussion.


\section{Conclusion}

** To be written **

Mexico's federal districts exhibit no party bias when analyzed at the state level. Big system responsiveness, typical of first-past-the-post system in units with few federal districts, is associated with the districts. So PRI's tendency to be over-represented more than PAN and PRD, and the PRI's winning of a couple extra seats with the redistricting proposal is not the product of partisan bias. If the PRI wins more this is due to its status as the largest party in more states that the other two major parties combined. 

\bibliographystyle{apsr}
%\bibliography{../../../../../mydocs/magar}
\bibliography{../bib/magar,../bib/redMex}

%% next command, in console, extracts only relevant paper references to extracted.bib (http://tex.stackexchange.com/questions/41821)
%bibexport -o extracted.bib myarticle.aux

\end{document}
