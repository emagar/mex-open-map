Two obstacles: that 50 percent vote is not the pivotal value that it is in balanced two-party competition; another the compositional nature of vote shares. 

Besides the meaningless .5 vote share threshold discussed in section \ref{S:bias}, the compositional nature of multi-party vote shares adds another layer of complexity. Unlike linear regression, the logit link in equation \ref{E:kingBi} complicates assessment of individual $\lambda$s' impact on seat shares. One common approach, comparative statics analysis through simulation (e.g., clarify)---i.e., letting one regressor of interest fluctuate while all others remain constant at illustrative values---is inapplicable to compositional multi-party votes. When $v_p$ fluctuates, the other vote shares cannot remain constant. Uniform swing models (Gudgin Taylor) overcome this complication by assuming votes to be won/lost relative to other parties' sizes. Instead of relying on such restrictive approach, we discuss estimated $\lambda$ magnitude and polarity, then assess their importance through swing ratios analysis of simulated elections---like Linzer does.

Unlike OLS coefficients, the logit link in our model impedes assessment of individual lambdas' impact on seat shares. One common approach (e.g., clarify) is comparative statics analysis, letting one regressor of interest fluctuate while all others remain constant at mean, mode, or other illustrative values. This approach is inapplicable to partisan bias in a multi-party setting, due to the compositional nature of vote shares (the regressors): when v_p fluctuates, all other vote shares do not remain constant. "Proportional swing" models (cites) remove this complication by assuming that votes are won/lost relative to other parties' sizes. Instead of relying on such restrictive approach, the revised manuscript proceeds like the original submission did: discussing lambda estimates' magnitude and polarity first, then assessing their importance through swing ratios analysis of simulated elections---like Linzer does. We have expanded *in text? footnote?*
