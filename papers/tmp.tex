The interpretation of observed total partisan bias volatility is not straightforward withoud the decomposition method.\footnote{While not as volatile as ours, partisan bias estimates in \citet{gelman.king.1994EvalElSysRedis} for the U.S.\ and \citet{jackmanMeasuringBias1994} for Australia also show inter-election drift.} After all, if partisan bias is systematic advantage conferred to some party, the ex-ante expectation is that, absent redistricting or a tectonic shock to the party system, the advantaged party should enjoy a more efficient conversion of votes into seats election after election. It should not, like the PRD's, shrink in presidential election years, or suddenly change polarity, like the PAN's in 2006. Decomposing the sources of bias sheds some light on the matter. The malapportionment component is more clearly associated than others with the stability expectation (or, at least, with a constant trend in the presence of creeping malapportionment): it originates squarely in institutions and deliberate, ex-ante human choices. Stability is harder to entertain for the turnout \citep[mobilization has an endogenous component,][]{cox.munger.1989,rosenstone.hansen.1993} and the distributive (distorsions can be deliberate, or not) components. Decomposition exposes the distributive as the key source of volatility.

