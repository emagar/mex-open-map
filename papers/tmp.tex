\subsection{Expectations}

If the board acts as power-sharing agent of the major parties, then we should observe the following: 

\begin{enumerate}

\item When a map is adopted, districts have no bias favoring one major party relative to another. If a new map has bias, it must be against minor parties relative to major ones. 

\item As maps age and district populations drift away from census relative levels, partisan bias (malapportionment-based) can creep in. partisan bias should therefore grow with each failure to redistrict. 

\item There is no partisan gerrymandering. Gerrymander-based advantage for a party one year will be compensated by gerrymander-based handicap in other years, changes from one year to next according to parties' electoral fortunes

\end{enumerate}




