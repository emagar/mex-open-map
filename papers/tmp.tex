We explore the independent contribution of these three sources of partisan bias in recent, multi-party Mexican lower-chamber federal legislative elections. Since democratizing in the second half of the 1990s, three major parties routinely win most votes, but up to 11 parties have fielded candidates. Leaving the compensatory proportional representation (PR, discussed below) seats aside, major parties have been systematically over-represented, jointly receiving 14 percent more single-member district (SMD) seats than votes on average. Individually, the former hegemonic Institutional Revolutionary Party (PRI) stands out, achieving remarkable 19 percentage points over-representation in the 2012 election with just 36 percent of the vote, and as many as 23 points in 2009 with a vote base of 39 percent. With such background, partisan bias merits investigation and analysis. 

Our method to achieve this builds upon work by \citet{grofman.etalBiasMalapp.1997}. Our contribution is three-fold. First, unlike \citeauthor{grofman.etalBiasMalapp.1997} (and unlike previous works---see footnote \ref{fn:cites}), our approach drops the restrictive assumption of two-party competition. National two-party systems remain exceptional even among plurality systems \citep{cox.1997}, so extending measurement to multi-party competition clears the way to test theoretical propositions using empirical data from numerous systems previously beyond reach. Second, we take often-ignored ``creeping malapportionment'' \citep{johnston.2002} into account. Malapportionment is most-often described as a deliberate choice to over-represent citizens residing in small-population districts and under-represent those in large population districts. Creeping malapportionment---notably prevalent in the United States prior to Supreme Court decisions in the 1960s---arises by the failure to redistrict using the most current population counts from a government census. Finally, we apply these advancements to examine the case of Mexican C\'amara de Diputados elections to assess our method in a multi-party setting. We uncover small, but systematic, partisan bias against the right relative to the country's former hegemonic ruling party, but especially relative to the left. Decomposition of bias into the three additive components reveals that the parts are often greater than the whole, contributing in opposing directions and, therefore, offsetting one another to a large extent. 

We explore the independent contribution of these three sources of partisan bias in multi-party systems. Our method builds upon work by \citet{grofman.etalBiasMalapp.1997}. Our contribution is three-fold. First, unlike \citeauthor{grofman.etalBiasMalapp.1997} (and unlike previous works---see footnote \ref{fn:cites}), our approach drops the restrictive assumption of two-party competition. National two-party systems remain exceptional even among plurality systems \citep{cox.1997}, so extending measurement to multi-party competition clears the way to test theoretical propositions using empirical data from numerous systems previously beyond reach. Second, we take often-ignored ``creeping malapportionment'' \citep{johnston.2002} into account. Malapportionment is most-often described as a deliberate choice to over-represent citizens residing in small-population districts and under-represent those in large population districts. Creeping malapportionment---notably prevalent in the United States prior to Supreme Court decisions in the 1960s---arises by the failure to redistrict using the most current population counts from a government census. Finally, we apply these advancements to examine the case of Mexican C\'amara de Diputados elections to assess our method in a multi-party setting. We uncover small, but systematic, partisan bias against the right relative to the country's former hegemonic ruling party, but especially relative to the left. Decomposition of bias into the three additive components reveals that the parts are often greater than the whole, contributing in opposing directions and, therefore, offsetting one another to a large extent. 

